\chapter{Analyse der Lernumgebung }  % Kapitel % Steht dann über dem Text
\label{chapter:Analyse der Lernumgebung}  % Steht als Text im Inhaltsverzeichnis
\index{Analyse der Lernumgebung} % für das Stichwortverzeichnis

%Hier wird dann der Text stehen. 
%oder Auflistungen;
%-----------------------------------------------------------------
Die zu unterweisende Lerngruppe besteht aus 6 Studierenden der Klasse CN3 der Staatlichen Technik Akademie Weilburg. Die Studierenden befinden sich im dritten Semester ihrer Ausbildung zum Staatlich geprüften Techniker der Computersystem und Netzwerktechnik. 
Die Studenten sind männlich, im Alter von 22 bis 32 Jahren.
Alle Studierenden verfügen über den Mittleren Bildungsabschluss, zum Teil auch über das Fachabitur oder das  Allgemeinbildente Abitur.
Die Studierenden haben Erfahrungen in Einzel- sowie Gruppenarbeit. 
Allgmeine fähigkeiten im Umgang mit dem PC und dem Internet können von allen zu Unterweisenden vorrausgesetzt werden.

