\chapter{Didaktische_Überlegungen}  % Kapitel % Steht dann über dem Text
\label{chapter:Didaktische_Überlegungen}  % Steht als Text im Inhaltsverzeichnis
\index{Didaktische_Üeberlegungen} % für das Stichwortverzeichnis

%Hier wird dann der Text stehen. 
%oder Auflistungen;
%--------------------------------------------------
\subsection{Gegenwarts und Zukunftsbedeutung}
Für die zu Unterweisenden ist es wichtig zu verstehen das diese Systeme zur erfassung und verwaltung von Problemen ein sehr wichtiger bestandteil beim finden von effizenten Problemlösungen ist. Durch diese Systeme können Computersystem und Netzwerktechniker mit einem berechenbaren Aufwand und mit einem hohen maß an Know How, Fehler bearbeiten. Das ist bei den heutigen und stetig wachsenen Anforderungen an die Technik und ihr Personal sehr wichtig. Alle Administratoren müssen sich deshalb zunehmend mit den grundlegenden Konzepten und Ideen die mit einem Trouble Ticket System verbunden sind auseinandersetzen. 

\subsection{Exemplarische Bedeutung}
Die Bedeutung einer funktionstüchtigen IT Infrastruktur ist für alle sehr gestiegen. Hier zu Lande kommt kein Unternehmen, ohne eine vernünftige EDV Umgebung, die zuverlässig Arbeitet aus. Wenn aber doch Probleme und Fehler auftreten, sollen diese möglichst schnell erfasst und bearbeitet werden. 
Um Fehler und Probleme erfassen zu können werden immer s.g. \" Trouble Ticket Systeme \" entwickelt. Diese Applikationen erfassen Probleme und verwalten die Lösungen, sie halten so die Komunikationswege kurz und vermeiden doppel arbeit. 
Unternehmen versprechen sich von Trouble Ticket Systemen, höhere Effizienz bei Problemlösungen. Sie haben daher auch eine hohe Wirtschaftliche bedeutung.


\subsection{Überprüfung}
Die Sensibilisierung mit wirklichkeitsgetreuen Beispielen, die selbst erarbeitet werden, sorgt für die nötige Objetivität und das Interesse an einem trouble ticket Systemen. 

