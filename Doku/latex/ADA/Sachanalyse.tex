\chapter{Sachanalyse}  % Kapitel % Steht dann über dem Text
\label{chapter:Sachanalyse}  % Steht als Text im Inhaltsverzeichnis
\index{Sachanalyse} % für das Stichwortverzeichnis

Ein Trouble Ticket lässt sich im Wesentlichen mit einer Karteikarte vergleichen. Bei erstmaligem Auftreten eines Problems wird diese Karteikarte neu angelegt. Jeder der das Problem bearbeitet trägt nun seine Diagnose, sowie die durchgeführten Handlungen ein und dokumentiert deren Erfolg. Die Karteikarte gibt nun einen schnellen Überblick, gewährleistet eine schnelle Einarbeitung und verhindert das gleiche Lösungswege mehrfach beschritten werden. Ist das Problem beseitigt wird die Karteikarte archiviert.
Ein solches System umfasst natürlich auch noch mehr Faktoren, meist ist eine Benutzerverwaltung und ein Dokumentationsarchive integriert.

Das jetzt neu an der Staatlichen Techniker Schule Weilburg eingeführte Trouble Ticket System \textit{Code Red} kann von jedem Internet Anschluss erreicht werden. Was bedeutet das diese Karteikarten ortsunabhängig geführt und bearbeitet werden können. Wenn man ein Benutzerkonto im System zu besitzt kann man von einem. 
