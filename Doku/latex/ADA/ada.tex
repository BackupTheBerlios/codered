%Kontext: ADA-Unterweisung für die praktische Prüfung am 29.05.06
%Changelog:
%Timestamp                      Name                      Äderungen und Begründung
%

%Dokumentklasse und einige globale Einstellungen
\documentclass[11pt,a4paper,titlepage,openright,multicol]{scrbook}

%Untersttzung für Zeichen außrhalb des ASCII-Zeichensatzes
\usepackage[utf8]{inputenc}
\usepackage{fontenc}

%Trennungsregeln
\usepackage[ngermanb]{babel}

\usepackage{multicol}

%zahlreiche Pakete und weitere Einstellungen
\usepackage{color}
\usepackage{fancyhdr}

\usepackage[a4paper,left=3cm,right=1.5cm,headheight=1.5cm]{geometry}
\usepackage{ngerman}
\usepackage[pdftex]{graphicx}

\usepackage[pdftex, bookmarks,
		colorlinks=true,
		linkcolor=black,
		urlcolor=black,
		pdftitle={CodeRed Unterweisung},
		pdfauthor={Marco Benecke, Jan Neuser},
		pdfsubject={Eine Unterweisung in die Software CodeRed},
		pdfkeywords={ADA, Unterweisung, Codered, STSWeilburg}]{hyperref}


\usepackage{longtable}

%Erstellung einer Indexdatei
\usepackage{makeidx}

\usepackage{textcomp}
\usepackage{verbatim}

%Verwendung von Font Type 1 fr bessere Lesbarkeit im Acrobat Reader
\usepackage{pslatex}

%Gliederungstiefe, Makropaket und Einstellung fr das Inhaltsverzeichnis
\usepackage{minitoc}
\setcounter{secnumdepth}{3}
\setcounter{tocdepth}{2}

%Aufnahme der Verzeichnisse (Stichwortverzeichnis, Abbildungs- und Tabellenverzeichnis) ins Inhaltsverzeichnis
\usepackage{tocbibind}

\usepackage{nomencl}

%\usepackage{listings}
%in der aktuellen Version des Styles fr listings haben sich einige �derungen ergeben, insbesondere wird label durch number ersetzt, das Stylefile wird nicht mehr ausgeliefert
%\lstset{captionpos=b, frame=trbl, frameround=ffff, labelstyle=\tiny, labelstep=5, firstlabel=1, labelsep=5pt}
%\lstset{captionpos=b, frame=trbl, frameround=ffff, numberstyle=\tiny, stepnumber=5, firstnumber=1, numbersep=5pt, aboveskip=10pt, belowskip=10pt}

% weitere Eigenschaften fr den Abstand des Listings
% aboveskip=10pt, belowskip=10pt

\usepackage{path}

% Das Paket parskip verhindert den Einzug am Anfang neuer Abs�ze und setzt einen sinnvollen Zeilenabstand zwischen den Abs�zen, wirkt sich darber hinaus auch Einrckungen in Aufz�lungen, Listen usw. aus.
\usepackage{parskip}

%sollte Fussnoten in Tabellen erm�lichen, berfordert aber Latex
%\usepackage{./texstyles/ftn}
%\ftn{tabular}

\makeatletter
%Breite fr die Seitennummerierung in Inhaltsverzeichnis
%Vermeidung des Fehlers Overfull \hbox
\renewcommand{\@pnumwidth}{2.5em}

%Abstand zwischen der Abschnittsnummer und der Kapitelberschrift im Inhaltsverzeichnis
\renewcommand*\l@section{\@dottedtocline{1}{1.0em}{2.3em}}
\renewcommand*\l@subsection{\@dottedtocline{2}{3.3em}{3.2em}}
\renewcommand*\l@subsubsection{\@dottedtocline{3}{6.5em}{4.1em}}
\renewcommand*\l@paragraph{\@dottedtocline{4}{9.5em}{5.0em}}
\renewcommand*\l@subparagraph{\@dottedtocline{5}{11.5em}{6.0em}}

%Kapitelnummerierung am Anfang jedes Buches, eingeleitet durch \part zurcksetzen
%Kompiliert jetzt zwar korrekt, verhindert aber ein vernnftiges Bookmarking durch das Paket hyperref
%\@addtoreset{chapter}{part}
\makeatother

%Seitennummerierung in arabischen Ziffern
%\pagenumbering{arabic}

\makeindex
\makeglossary

%Anfang des Dokuments
\begin{document}

%Einleitung mit eigenen Seitennummerierung
%\frontmatter

%Hauptteil
%\mainmatter

%Titelseite
%Kontext: Komplettes Handbuch für Codered
%Changelog:
%Timestamp			Name			Äderungen und Begründung
%

\begin{titlepage}

\title{Einführung in das Ticketsystem \textit{CodeRed} \\ - Am Beispiel: Benutzerprofil anlegen -}
\author{Marco Benecke \and Jan Neuser}
\date{Ausgabe 0.1 vom \today}

\publishers{Staatliche Techniker Schule Weilburg}

\uppertitleback{

\textbf{Autoren:} \\
Benecke, Marco erreichbar unter \textit{benecke@gmail.com} \\
Neuser, Jan mit \textit{jan@truematrix.de} \\

\textbf{Korrekturleser:} \\
 \\ \\
 \\ \\
}

\lowertitleback{Dieses Buch unterliegt den Grundgedanken der GPL und wird ausschließich mit Hilfe des Satzsystems {\LaTeX} unter Linux geschrieben. Die Veröffentlichung erfolgt als plattformunabhägiges PDF. Vielen Dank.}

\maketitle

\end{titlepage}



%Inhaltsverzeichnis
\tableofcontents

\part{Unterweisung}
\label{part:Unterweisung}
%\input{einleitung}


\chapter{Analyse der Lernumgebung }  % Kapitel % Steht dann über dem Text
\label{chapter:Analyse der Lernumgebung}  % Steht als Text im Inhaltsverzeichnis
\index{Analyse der Lernumgebung} % für das Stichwortverzeichnis

%Hier wird dann der Text stehen. 
%oder Auflistungen;
%-----------------------------------------------------------------
Die zu unterweisende Lerngruppe besteht aus 6 Studierenden der Klasse CN3 der Staatlichen Technik Akademie Weilburg. Die Studierenden befinden sich im dritten Semester ihrer Ausbildung zum Staatlich geprüften Techniker der Computersystem und Netzwerktechnik. 
Die Studenten sind männlich, im Alter von 22 bis 32 Jahren.
Alle Studierenden verfügen über den Mittleren Bildungsabschluss, zum Teil auch über das Fachabitur oder das  Allgemeinbildente Abitur.
Die Studierenden haben Erfahrungen in Einzel- sowie Gruppenarbeit. 
Allgmeine fähigkeiten im Umgang mit dem PC und dem Internet können von allen zu Unterweisenden vorrausgesetzt werden.


\chapter{Sachanalyse}  % Kapitel % Steht dann über dem Text
\label{chapter:Sachanalyse}  % Steht als Text im Inhaltsverzeichnis
\index{Sachanalyse} % für das Stichwortverzeichnis

Ein Trouble Ticket lässt sich im Wesentlichen mit einer Karteikarte vergleichen. Bei erstmaligem Auftreten eines Problems wird diese Karteikarte neu angelegt. Jeder der das Problem bearbeitet trägt nun seine Diagnose, sowie die durchgeführten Handlungen ein und dokumentiert deren Erfolg. Die Karteikarte gibt nun einen schnellen Überblick, gewährleistet eine schnelle Einarbeitung und verhindert das gleiche Lösungswege mehrfach beschritten werden. Ist das Problem beseitigt wird die Karteikarte archiviert.
Ein solches System umfasst natürlich auch noch mehr Faktoren, meist ist eine Benutzerverwaltung und ein Dokumentationsarchive integriert.

Das jetzt neu an der Staatlichen Techniker Schule Weilburg eingeführte Trouble Ticket System \textit{Code Red} kann von jedem Internet Anschluss erreicht werden. Was bedeutet das diese Karteikarten ortsunabhängig geführt und bearbeitet werden können. Wenn man ein Benutzerkonto im System zu besitzt kann man von einem. 





\part{Kopiervorlagen}
\label{part:Kopiervorlagen}

%Anhang
%\backmatter
\appendix
\part{Anhang}
\label{part:Anhang}

%Entwurf von Tobias Mucke
%Erg�zungen von Michael Petter
%Kontext: SuSE 8.0
%Changelog:
%Timestamp                       Name                       �derungen und Begrndung
%26. Oktober 2002           Tobias Mucke        erweitert und korrigiert

\chapter{Literaturverzeichnis}
\begin{description}
\item{[1]}
Marco Benecke: \textit{Linux Installation, Konfiguration, Anwendung},
noch in planung :)

\end{description}






\printglossary

%Abbbildungsverzeichnis
\listoffigures

%Tabellenverzeichnis
\listoftables

%Listingverzeichnis
%\lstlistoflistings

% Stichwortverzeichnis
\printindex

%Ende des Dokuments
\end{document}
