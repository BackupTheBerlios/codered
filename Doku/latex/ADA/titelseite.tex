%Kontext: Komplettes Handbuch für Codered
%Changelog:
%Timestamp			Name			Äderungen und Begründung
%

\begin{titlepage}

\title{Einführung in das Konzept eines \mbox{\textit{Trouble Ticket Systems}} \\ - Am Beispiel der STSWeilburg  -}
\author{Marco Benecke \and Jan Neuser}
\date{Ausgabe 0.1 vom \today}

\publishers{Staatliche Techniker Schule Weilburg}

\uppertitleback{

\textbf{Versicherung:} \\
Ich/wir versichere/n den hier vorgelegten Entwurf einer Unterweisung eigen- und selbstständig nur unter Verwendung der angegeben Hilfsmittel und Quellen erstellt zu haben. \\
\begin{center}
% use packages: array
\begin{tabular}{ll}
Thema: & Einführung in das Konzept eines \textit{Trouble Ticket systems} - Am Beispiel der STSWeilburg\\
Datum: &  29.03.06\\ 
Methoden: &  \\ 
Ort/Raum: & Weilburg/R53 \\ 
Dauer: & 45 min \\ 
Zusätzliche Quellen: & siehe Anlage \\
\end{tabular}
\end{center}
\vspace{2cm}
\textbf{Signaturen}:\\
\vspace{2cm}

\begin{center}
	\begin{minipage}[t]{4cm}
		\begin{tabular}{p{4cm}}
		\hline\\
		Marco Benecke
		\end{tabular}
	\end{minipage} 
\hspace{2cm}
	\begin{minipage}[t]{4cm}
		\begin{tabular}{p{4cm}}
		\hline\\
		Jan Neuser
		\end{tabular}
	\end{minipage}
\end{center}

\vspace{2cm}

\textbf{Autoren:} \\
Benecke, Marco erreichbar unter \textit{benecke@gmail.com} \\
Neuser, Jan mit \textit{jan@truematrix.de} \\

\vspace{1cm}

\textbf{Korrekturleser:} \\
 \\
 \\
}

\lowertitleback{Dieses Buch unterliegt den Grundgedanken der GPL und wird ausschließich mit Hilfe des Satzsystems {\LaTeX} unter Linux geschrieben. Die Veröffentlichung erfolgt als plattformunabhägiges PDF. Vielen Dank.}

\maketitle

\end{titlepage}

