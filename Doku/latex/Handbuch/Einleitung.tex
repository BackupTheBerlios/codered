\chapter{Die Idee}  % Kapitel % Steht dann über dem Text
\label{chapter:Die Idee}  % Steht als Text im Inhaltsverzeichnis
\index{Die Idee} % für das Stichwortverzeichnis

Die Staatliche Technikerschule Weilburg (STSW)
übernimmt seit Jahren verschiedene IT Projekte
von Firmen und Schulen im Rahmen der praktischen
Abschlussprüfungen der angehenden Techniker. \\
\\
Nun stellte sich heraus, dass mit der Zeit
Wartungsarbeiten an den Projekten anfallen oder
dass Defekte behoben werden müssen.
Die Schulen, die selbst keine Ressourcen für die
Problembehebung bündeln konnten, haben sich
wieder an die STSW gewannt, um Hilfe bei ihren
Problemen zu bekommen. \\
\\
Diese Probleme wurden an einen Lehrer weitergeleitet,
dieser löste diese entweder selbst oder
beauftragte einige Studierende mit der Lösung.
Schwierigkeiten bestanden darin, dass meist nur noch die
Dokumentation des Projektes zur Lösungshilfe bereit
steht. Alle Änderungen und evtl. schon vorhandene
gelöste Probleme sind nicht dokumentiert gewesen und gingen
somit immer mit dem Abschluss der Studierenden verloren.\\
\\
Die Lösung für diese vielen Probleme wurde unter dem Namen \textbf{CodeRed} 
entwickelt. Ca ein Jahr hat sich das Team gedanken über eine Lösung und 
deren umsetzung gedanken gemacht. CodeRed verwaltet zentral alle Probleme 
die bei den Klienten entstehen. Das System ist webbasierend und dadurch von jedem 
Anwender mit einem Internetzugang anwendbar. Serviceteams, Clienten, Kontakte oder auch Mentoren können über die Datenbankapplication schnell und einfach ihre Probleme oder Lösungen dokumentieren. \\
\\
Bei der Entwicklung wurde besonders darauf geachtet, die komplexen Abläufe die Auftreten können, klar und einfach zu halten. Niemand sollte eine gößere Einführung in die Software benötigen um sie Nutzen zu können. Deswegen wurde auf neue Technologien die unter dem Begriff Web2.0 in den Fachmedien bekannt geworden sind, großen Wert gelegt.\\
\\
Wir die Entwickler hoffen eine leistungfähige Lösung entwickelt zu haben, die von jedermann bedient werden kann.   
