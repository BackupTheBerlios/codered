\chapter{Vorausetzungen }  % Kapitel % Steht dann über dem Text
\label{chapter:Vorausetzungen}  % Steht als Text im Inhaltsverzeichnis
\index{Vorausetzungen} % für das Stichwortverzeichnis

Für eine erfolgreiche Installation eines Servers müssen verschiedene Hardware und Software Vorrausetzungen erfüllt werden. Die folgen Auflistungen beinhalten nur die Groben Hardware bedingungen und die Hauptsoftware Packete. Bei der spätern Installationsanleitungen werden noch verschiedene kleiner Packete installiert, um einen Reibunslosen Betrieb zu gewährleisten. \\
\section{Hardware Vorrausetzungen}
\label{section:Hardware Vorraussetzungen}
\index{Hardware}
\index{Hardware}
\begin{itemize}
\item Intel x86-basierend System 
\item Schnittstellen: 1x LAN
\item Arbeitsspeicher: min 256 MB \footnotemark[1]
\item Laufwerke: 1x DVD Laufwerk
\item HDD: Je nach bedarf! (Aktuell 40 Gb)
\end{itemize}
\footnote[1]{Die Anforderungen an den Arbeitsspeicher steigen schnell, da grade Datenbank Anwendungen viel Arbeitsspeicher benötigen.}

\section{Software Vorrausetzungen}
\label{section:Software Vorraussetzungen}
\index{Software}
\index{Software}
\begin{itemize}
\item Linux Distributions DVD Debian ab Version 3.1 - \href{http://www.debian.org}{http://www.debian.org}
\item Apache 2 Webserver mit Fast CGI -
\href{http://www.apache2.org}{http://www.apache2.org}
\item MySQL Server ab Version 4 - 
\href{http://dev.mysql.com}{http://www.mysql.com}
\item Ruby installations Packet mit Gems Packetverwaltung - \href{http://www.ruby-lang.org/en/}{http://www.ruby-lang.org/en}
\item Ruby on Rails Framework -
\href{http://www.rubyonrails.com/}{http://www.rubyonrails.com}
\end{itemize}
