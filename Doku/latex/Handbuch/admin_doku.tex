%Kontext: Admin Dokumentation zum Projekt CodeRed 
%Changelog:
%Timestamp                      Name                      Äderungen und Begründung
%

%Dokumentklasse und einige globale Einstellungen
\documentclass[11pt,a4paper,titlepage,openright,multicol]{scrbook}

%Untersttzung für Zeichen außrhalb des ASCII-Zeichensatzes
\usepackage[utf8]{inputenc}
\usepackage{fontenc}

%Trennungsregeln
\usepackage[ngermanb]{babel}

\usepackage{multicol}

%zahlreiche Pakete und weitere Einstellungen
\usepackage{color}
\usepackage{fancyhdr}

%\usepackage[a4paper,left=3cm,right=1.5cm,headheight=1.5cm]{geometry}
\usepackage{ngerman}
\usepackage[pdftex]{graphicx}

\usepackage[pdftex, bookmarks,
		colorlinks=true,
		linkcolor=black,
		urlcolor=black,
		pdftitle={CodeRed Unterweisung},
		pdfauthor={Marco Benecke, Jan Neuser},
		pdfsubject={Eine Unterweisung in die Software CodeRed}
		pdfkeywords={ADA, Unterweisung, Codered, STSWeilburg}]{hyperref}


\usepackage{longtable}

%Erstellung einer Indexdatei
\usepackage{makeidx}

\usepackage{textcomp}
\usepackage{verbatim}

%Verwendung von Font Type 1 fr bessere Lesbarkeit im Acrobat Reader
\usepackage{pslatex}

%Gliederungstiefe, Makropaket und Einstellung fr das Inhaltsverzeichnis
\usepackage{minitoc}
\setcounter{secnumdepth}{3}
\setcounter{tocdepth}{2}

%Aufnahme der Verzeichnisse (Stichwortverzeichnis, Abbildungs- und Tabellenverzeichnis) ins Inhaltsverzeichnis
\usepackage{tocbibind}

\usepackage{nomencl}

\usepackage{listings}
%in der aktuellen Version des Styles fr listings haben sich einige �derungen ergeben, insbesondere wird label durch number ersetzt, das Stylefile wird nicht mehr ausgeliefert
%\lstset{captionpos=b, frame=trbl, frameround=ffff, labelstyle=\tiny, labelstep=5, firstlabel=1, labelsep=5pt}
%\lstset{captionpos=b, frame=trbl, frameround=ffff, numberstyle=\tiny, stepnumber=5, firstnumber=1, numbersep=5pt, aboveskip=10pt, belowskip=10pt}

% weitere Eigenschaften fr den Abstand des Listings
% aboveskip=10pt, belowskip=10pt

\usepackage{path}

% Das Paket parskip verhindert den Einzug am Anfang neuer Abs�ze und setzt einen sinnvollen Zeilenabstand zwischen den Abs�zen, wirkt sich darber hinaus auch Einrckungen in Aufz�lungen, Listen usw. aus.
\usepackage{parskip}

%sollte Fussnoten in Tabellen erm�lichen, berfordert aber Latex
%\usepackage{./texstyles/ftn}
%\ftn{tabular}

\makeatletter
%Breite fr die Seitennummerierung in Inhaltsverzeichnis
%Vermeidung des Fehlers Overfull \hbox
\renewcommand{\@pnumwidth}{2.5em}

%Abstand zwischen der Abschnittsnummer und der Kapitelberschrift im Inhaltsverzeichnis
\renewcommand*\l@section{\@dottedtocline{1}{1.0em}{2.3em}}
\renewcommand*\l@subsection{\@dottedtocline{2}{3.3em}{3.2em}}
\renewcommand*\l@subsubsection{\@dottedtocline{3}{6.5em}{4.1em}}
\renewcommand*\l@paragraph{\@dottedtocline{4}{9.5em}{5.0em}}
\renewcommand*\l@subparagraph{\@dottedtocline{5}{11.5em}{6.0em}}

%Kapitelnummerierung am Anfang jedes Buches, eingeleitet durch \part zurcksetzen
%Kompiliert jetzt zwar korrekt, verhindert aber ein vernnftiges Bookmarking durch das Paket hyperref
\@addtoreset{chapter}{part}
\makeatother

%Seitennummerierung in arabischen Ziffern
\pagenumbering{arabic}

\makeindex
\makeglossary

%Anfang des Dokuments
\begin{document}

%Einleitung mit eigenen Seitennummerierung
\frontmatter

%Hauptteil
\mainmatter

%Titelseite
\input{titelseite_admin}

%Inhaltsverzeichnis
\tableofcontents

\part{Einleitung}
\label{part:Einleitung}
\chapter{Die Idee}  % Kapitel % Steht dann über dem Text
\label{chapter:Die Idee}  % Steht als Text im Inhaltsverzeichnis
\index{Die Idee} % für das Stichwortverzeichnis

Die Staatliche Technikerschule Weilburg (STSW)
übernimmt seit Jahren verschiedene IT Projekte
von Firmen und Schulen im Rahmen der praktischen
Abschlussprüfungen der angehenden Techniker. \\
\\
Nun stellte sich heraus, dass mit der Zeit
Wartungsarbeiten an den Projekten anfallen oder
dass Defekte behoben werden müssen.
Die Schulen, die selbst keine Ressourcen für die
Problembehebung bündeln konnten, haben sich
wieder an die STSW gewannt, um Hilfe bei ihren
Problemen zu bekommen. \\
\\
Diese Probleme wurden an einen Lehrer weitergeleitet,
dieser löste diese entweder selbst oder
beauftragte einige Studierende mit der Lösung.
Schwierigkeiten bestanden darin, dass meist nur noch die
Dokumentation des Projektes zur Lösungshilfe bereit
steht. Alle Änderungen und evtl. schon vorhandene
gelöste Probleme sind nicht dokumentiert gewesen und gingen
somit immer mit dem Abschluss der Studierenden verloren.\\
\\
Die Lösung für diese vielen Probleme wurde unter dem Namen \textbf{CodeRed} 
entwickelt. Ca ein Jahr hat sich das Team gedanken über eine Lösung und 
deren umsetzung gedanken gemacht. CodeRed verwaltet zentral alle Probleme 
die bei den Klienten entstehen. Das System ist webbasierend und dadurch von jedem 
Anwender mit einem Internetzugang anwendbar. Serviceteams, Clienten, Kontakte oder auch Mentoren können über die Datenbankapplication schnell und einfach ihre Probleme oder Lösungen dokumentieren. \\
\\
Bei der Entwicklung wurde besonders darauf geachtet, die komplexen Abläufe die Auftreten können, klar und einfach zu halten. Niemand sollte eine gößere Einführung in die Software benötigen um sie Nutzen zu können. Deswegen wurde auf neue Technologien die unter dem Begriff Web2.0 in den Fachmedien bekannt geworden sind, großen Wert gelegt.\\
\\
Wir die Entwickler hoffen eine leistungfähige Lösung entwickelt zu haben, die von jedermann bedient werden kann.   

\chapter{Danksagung}  % Kapitel % Steht dann über dem Text
\label{chapter:Danksagung}  % Steht als Text im Inhaltsverzeichnis
\index{Danksagung} % für das Stichwortverzeichnis

Das Projekt CodeRed konnte nur durch die Unterstützung von vielen verschieden Personen und Freien Internet Diensten verwirklicht werden. So möchten wir uns bei allen Menschen und Institutionen für ihre Ideen, Kretik und ihr Lob bedanken. \\
Ein paar Vorbilder und Hilfen für die Umsetzung dieses Projektes waren: 
\\
\textbf{Dienste und Pattformen}
\begin{itemize}
\item \href{http://www.berlios.de}{BerliOS}, die jedem Entwickeler eine sehr gute freie Kommunikatiosplattform zur verfügung stellt,
\item \href{http://www.google.de}{Google}, mit ihren Entwicklungen wie Gmail waren sie ein sehr großes Vorbild in Sachen Anwenderfreundlichkeit. 
\item \href{http://rubyonrails.de}{ROR Entwickler Deutschland}, mit ihrer Hilfe über die Mailingliste haben sie uns über einige Fehler hinweggeholfen
\end{itemize}
\chapter{Entwickler}  % Kapitel % Steht dann über dem Text
\label{chapter:Entwickler}  % Steht als Text im Inhaltsverzeichnis
\index{Entwickler} % für das Stichwortverzeichnis

Als Entwickeler von CoderRed möchten wir uns gerne hier kurz vorstellen, wir stehen allen die weitere Fragen zu diesem Projekt haben natürlich gerne zur Verfügung. \\
\\
Allgemein besteht das Entwickelerteam von CodeRed aus zwei Studierenden der Staatlichen Techniker Schule Weilburg. Wir haben im Jahr 2004 die Ausbildung zum Staatlichen Techniker der Informationstechnik, Schwerpunkt Computersystem und Netzwerktechnik begonnen und Sommer 2005 dieses Projekt übernommen. Für uns stand bei der Übernahme der Aufgabe vorallem eines im Vordergrung, wir wollen unbedingt etwas machen was wir noch nicht beherschen und wir viel Lernen könnten. So war die Entwicklung eines Webbasierenden Trouble Ticket System eine nahe zu ideale Aufgabe, auch wenn wir erst nicht wussten wie wir diese Aufgabe lösen sollten.   
\\
\\
Backend Entwickeler \\
\textbf{Marco Benecke}
\\
\begin{picture}(150,200)
\includegraphics{../bilder/marcob.jpg}
\end{picture}
\\
Marco Benecke \\
Bitzengarten 16 \\
35614 Asslar-Oberlemp \\
\\
\textbf{Konatkt} \\
\begin{itemize}
\item Mobil: 0175 - 74 10 565
\item E-Mail: benecke@gmail.com
\item JabberID: Risktaker@jabber.org 
\item Homepage: \href{http://www.bytes-delivery.de}{bytes-delivery.de}
\end{itemize}
\textbf{Werdegang} \\
\begin{itemize}
\item Realschulabschluss auf der Gesamtschule Asslar-Hermannstein
\item Ausbildung zum Energieelektroniker-Betriebstechnik bei der Buderus Guss GmbH in Wetzlar
\item 5./Technische Schule der Luftwaffe in Erndtebrück Staabsgebiet 6 – Netzbetreuer und Administrator
\item Weiterbildung zum Staatlich geprüften Techniker in Weilburg, Schwerpunkt: Computersystem und Netzwerktechnik	
\end{itemize}
FrontEnd Entwickeler \\
\textbf{Jan Neuser}
\\
\begin{picture}(150,200)
\includegraphics{../bilder/marcob.jpg}
\end{picture}
\\
Jan Neuser \\
Münchbornstr.4 \\
35753 Greifenstein Arborn \\
\\
\textbf{Konatkt} \\
\begin{itemize}
\item Mobil: 0177 - 60 39 783
\item E-Mail: jan@truematrix.de
\item JabberID: nean77@gmail.com 
\item Homepage: \href{http://www.truematrix.de}{truematrix.de}
\end{itemize}
\textbf{Werdegang} \\
\begin{itemize}
\item Realschulabschluss auf der Gesamtschule Driedorf
\item Ausbildung zum IT- Systemelektroniker in Sinn bei der Firma Dietermann \and Heuser GmbH  (Heute: Hees Bürowelt)
\item Stabskompanie Panzerbrigade 14, Stabsdienst - Truppenverwaltung
\item Weiterbildung zum Staatlich geprüften Techniker in Weilburg, Schwerpunkt: Computersystem und Netzwerktechnik	
\end{itemize}



\chapter{Lizenzen}  % Kapitel % Steht dann über dem Text
\label{chapter:Lizenzen}  % Steht als Text im Inhaltsverzeichnis
\index{Lizenzen} % für das Stichwortverzeichnis

\begin{figure}[h]
\begin{center}
   \includegraphics[width=130pt]{../bilder/gnu-head-sm.jpg}
   \caption{GNU Projekt Logo}
   \label{GPL Lizenz}
\end{center}
\end{figure}
Das Projekt CodeRed unterliegt in seinen Grundzügen der \href{http://de.wikipedia.org/wiki/GPL}{GPL Lizenz} und wurde mit Hilfe von Freien Software Produkten und Entwicklungsumgebungen geschaffen.
\begin{figure}[h]
\begin{center}
   \includegraphics[width=90pt]{../bilder/mysql.jpg}
   \includegraphics[width=90pt]{../bilder/rails.png}
   \includegraphics[width=90pt]{../bilder/debian.png}
   \includegraphics[width=90pt]{../bilder/apache.png}
   \caption{GNU Projekt Logo}
   \label{GPL Lizenz}
\end{center}
\end{figure}

\part{Vorbereitungen}
\label{part:Vorbereitungen}
\input{Zweck}
\chapter{Vorausetzungen }  % Kapitel % Steht dann über dem Text
\label{chapter:Vorausetzungen}  % Steht als Text im Inhaltsverzeichnis
\index{Vorausetzungen} % für das Stichwortverzeichnis

Für eine erfolgreiche Installation eines Servers müssen verschiedene Hardware und Software Vorrausetzungen erfüllt werden. Die folgen Auflistungen beinhalten nur die Groben Hardware bedingungen und die Hauptsoftware Packete. Bei der spätern Installationsanleitungen werden noch verschiedene kleiner Packete installiert, um einen Reibunslosen Betrieb zu gewährleisten. \\
\section{Hardware Vorrausetzungen}
\label{section:Hardware Vorraussetzungen}
\index{Hardware}
\index{Hardware}
\begin{itemize}
\item Intel x86-basierend System 
\item Schnittstellen: 1x LAN
\item Arbeitsspeicher: min 256 MB \footnotemark[1]
\item Laufwerke: 1x DVD Laufwerk
\item HDD: Je nach bedarf! (Aktuell 40 Gb)
\end{itemize}
\footnote[1]{Die Anforderungen an den Arbeitsspeicher steigen schnell, da grade Datenbank Anwendungen viel Arbeitsspeicher benötigen.}

\section{Software Vorrausetzungen}
\label{section:Software Vorraussetzungen}
\index{Software}
\index{Software}
\begin{itemize}
\item Linux Distributions DVD Debian ab Version 3.1 - \href{http://www.debian.org}{http://www.debian.org}
\item Apache 2 Webserver mit Fast CGI -
\href{http://www.apache2.org}{http://www.apache2.org}
\item MySQL Server ab Version 4 - 
\href{http://dev.mysql.com}{http://www.mysql.com}
\item Ruby installations Packet mit Gems Packetverwaltung - \href{http://www.ruby-lang.org/en/}{http://www.ruby-lang.org/en}
\item Ruby on Rails Framework -
\href{http://www.rubyonrails.com/}{http://www.rubyonrails.com}
\end{itemize}

\chapter{Zeitplan}  % Kapitel % Steht dann über dem Text
\label{chapter:Zeitplan}  % Steht als Text im Inhaltsverzeichnis
\index{Zeitplan} % für das Stichwortverzeichnis
\begin{figure}[h]
\begin{center}
   \includegraphics[width=5cm]{../bilder/Zeitplaninstall.png}
   \caption{Zeitplan Systeminstallation}
   \label{Zeitplan-Ablauf}
\end{center}
\end{figure}

Die Grafik zeigt den ideal Ablauf einer Serverinstallation für einen CodeRed Server. Probleme mit der Hardware oder auch mit der Anbindung sind hier nicht berücksichtigt. Die Intallation kann sich bei Problemen, die nichts direkt mit dem eigentlichen System zutun haben, um eine nicht bestimmbare Zeit verlängern.




\part{Der Server}
\label{part:Der Server}
\chapter{Basis Installation}  % Kapitel % Steht dann über dem Text
\label{Basis Installation}  % Steht als Text im Inhaltsverzeichnis
\index{Basis Installation} % für das Stichwortverzeichnis

Vorstellung unsers Ausgangsserver
Hardware , Photo 

\section{BIOS Einstellungen}
\label{section:BIOS Einstellungen}
\index{BIOS Einstellungen}
\index{BIOS Einstellungen}

%Text Text

\section{Debian Grundsystem}
\label{section:Debian Grundsystem}
\index{Basis Installation Debian}
\index{Debian Grundsystem}

%Text Text

\section{Debian Grundkonfiguration}
\label{section:Debian Grundkonfiguration}
\index{Konfiguration des Grunssystems}
\index{Debian Grundkonfiguration}



\input{pro_system} 

\part{Dienste}
\label{part:Dienste}
\chapter{DynDNS Dienst}  % Kapitel % Steht dann über dem Text
\label{chapter:DynDNS Dienst}  % Steht als Text im Inhaltsverzeichnis
\index{DynDNS Dienst} % für das Stichwortverzeichnis

\begin{figure}[h]
\begin{center}
   \includegraphics[width=200pt]{../bilder/dyndns.png}
   \caption{Dyndns.org}
   \label{DynDNS.org}
\end{center}
\end{figure}
\textbf{Erläuterung zu DynDNS:} \\
Ein DynDNS- oder dynamischer Domain-Name-System-Eintrag bewirkt, dass ein Rechner, der eine wechselnde IP-Adresse besitzt, immer über den selben Domainnamen angesprochen werden kann.\\
\\
Ständig wechselnde Einträge sind im Domain Name System eigentlich nicht vorgesehen, stattdessen sollen Netzressourcen gespart werden, indem Einträge – oft mehrere Stunden oder sogar Tage – zwischengespeichert werden. Um nun dynamische DNS-Einträge zu ermöglichen, wird die Zeit, wie lange der Eintrag zwischengespeichert werden soll, auf das erlaubte Minimum von 60 Sekunden gesetzt.\\
\\
Um einen DynDNS-Eintrag in den Nameservern des Betreibers zu aktualisieren, wird üblicherweise ein DynDNS-Client installiert. Dies ist ein Programm, das sich automatisch bei einem IP-Wechsel mit dem DynDNS-Server verbindet und seine neue IP-Adresse übermittelt.
\\
Quelle: \href{http://de.Wikipedia.org/wiki/DynDNS}{Wikipedia: DynDNS} \\
\\
\textbf{Warum setzen wir DynDNS ein:}
DynDNS wird eingesetzt durch die etwas erschwerten Gegebenheiten beim Kunden. Der Kunde besitzt leider keine feste IP Adresse vorort und hat auch keine ausreichenden Zugriffsrechte auf seine vorhanden Domänen.\\
\\
\textbf{http://www.dyndns.com/ - Account}
\begin{table}[htbp]
\begin{center}
\begin{tabular*}{0.95\textwidth}{p{0.3\textwidth}p{0.6\textwidth}}
\hline
\textbf{Frage?} & \textbf{Auswahl} \\
\hline
Email: & CodeRed@farbspielchen.de \\
Name: & stsw \\
Passwort: & \textbf{2KkDCrcB3Q} \\
Hostname: & stsw-intern.dyndns.org \\
\hline
\end{tabular*}
\caption{DynDNS Account}
\label{table:DynDNS Account}
\end{center}
\end{table}


\chapter{Mail Dienst}  % Kapitel % Steht dann über dem Text
\label{chapter:Mail Dienst}  % Steht als Text im Inhaltsverzeichnis
\index{Mail Dienst} % für das Stichwortverzeichnis




\input{backup}

\part{Future List}
\label{part:Future List}
%\input{future}

%Anhang
\backmatter
\appendix
\part{Anhang}
\label{part:Anhang}

%Entwurf von Tobias Mucke
%Erg�zungen von Michael Petter
%Kontext: SuSE 8.0
%Changelog:
%Timestamp                       Name                       �derungen und Begrndung
%26. Oktober 2002           Tobias Mucke        erweitert und korrigiert

\chapter{Literaturverzeichnis}
\begin{description}
\item{[1]}
Marco Benecke: \textit{Linux Installation, Konfiguration, Anwendung},
noch in planung :)

\end{description}






\printglossary

%Abbbildungsverzeichnis
\listoffigures

%Tabellenverzeichnis
\listoftables

%Listingverzeichnis
\lstlistoflistings

% Stichwortverzeichnis
\printindex

%Ende des Dokuments
\end{document}
