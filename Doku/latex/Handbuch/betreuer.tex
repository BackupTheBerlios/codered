\chapter{Betreuer}  % Kapitel % Steht dann über dem Text
\label{chapter:Betreuer}  % Steht als Text im Inhaltsverzeichnis
\index{Betreuer} % für das Stichwortverzeichnis

\textbf{Die Betreuer},sind im Sinne des CodeRed Systems die Arbeiter. Betreuer sollen freiwillige Studierende der STSW werden. Grundsätzlich kann aber jedes Mitglied des Systems zu einem Betreuer hoch gestuft werden.\\
Die Betreuer werden von den Mentoren beauftragt Tickets zu bearbeiten und zu dokumentieren. Sie selbst dürfen keine Ticket zuweisen. Das Erstellen von Tickets ist ihnen aber möglich. \\
Sollte es vorkommen das ein Betreuer aus verschieden Gründen das Ticket nicht bearbeiten kann, darf er das ihm zugewiesene Ticket mit einer Erklärung zurückgeben. Ein Mentor kann dann das Ticket jemanden neu zuweisen. 
\\
\\
\textbf{Die Möglichkeiten von Betreuern im CodeRed System:}
\begin{figure}[h]
\begin{center}
   \includegraphics[width=400pt]{../bilder/betreuer.png}
   \caption{Systemstruktur- Betreuer}
   \label{Systemstruktur - Betreuer}
\end{center}
\end{figure}
\\
Der Betreuer hat die Pflicht das vom Mentoren zugewiesene Ticket, nach besten Wissen zu bearbeiten. Alle Arbeiten oder Gespräche über das Ticket sollen von Betreuer in Workflows festgehalten werden. Sollte es ihm nicht möglich das Ticket zu bearbeiten, hat er jederzeit die Möglichkeit das Ticket zurückzugeben. Der Betreuer hat Zugriff auf alle Clienten und kann die vorhanden Dokumentationen der Clienten herunterladen oder/und auch eigene angefertigte Dokumentationen, zu dem jeweiligen Clienten, uploaden.

