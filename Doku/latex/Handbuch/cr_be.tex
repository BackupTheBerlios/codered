\chapter{Was macht CodeRed?}  % Kapitel % Steht dann über dem Text
\label{chapter:Was macht CodeRed?}  % Steht als Text im Inhaltsverzeichnis
\index{Was macht CodeRed?} % für das Stichwortverzeichnis
\begin{figure}[h]
\begin{center}
   \includegraphics[width=150pt]{../bilder/crlogo.png}
   \caption{CodeRed Logo}
   \label{CodeRed Logo}
\end{center}
\end{figure}
In diesem Abschnitt soll kurz die grundlegende Idee, die hinter Trouble Tickets im Allgemeinen und Trouble Ticket Systemen im Speziellen steht, erläutert werden. An einem kleinen Beispiel wird gezeigt,wofür Trouble Ticket Systeme in der Praxis verwendet werden können und wo die Vorteile dieser Systeme liegen.\\
\\
Das folgende Beispiel soll verdeutlichen, was ein Trouble Ticket System ist und wie damit in der Praxis Zeit und Geld eingespart werden können.
Nehmen wir an, dass Max Mustermann Fabrikant ist und Videorecorder produziert. Da die
Programmierung der Videorecorder sehr unübersichtlich und kompliziert ist, wenden sich die Kunden von Herrn Mustermann gerne und häufig mit Supportanfragen per Mail an ihn. An manchen Tagen kann Herr Mustermann der Mailflut kaum Herr werden und so kommt es, dass seine Kunden sich einige Zeit gedulden müssen, bis die Antwort mit der rettenden Lösung bei ihnen eintrifft. Manchen Kunden dauert dies jedoch zu lange, eine weitere Email mit dem gleichen Inhalt wird an Herrn Mustermann geschickt.
Die Emails mit den Supportanfragen werden alle in eine INBOX weitergeleitet, wie sie von fast allen Emailprogrammen verwendet wird.
An manchen Tagen ist die Anfragewelle besonders groß und Herr Mustermann sieht sich außerstande, alle Mails noch in einem vertretbaren Zeitrahmen zu beantworten. Aus diesem Grund kommandiert er seine Entwickler Meier und Schulze zur Bearbeitung der Supportanfragen ab. Da von allen das gleiche System benutzt wird, greifen alle auf die gleiche INBOX und daher auch auf die gleichen Mails zu.Meier und Schulze haben jedoch keine Ahnung, dass manch ein Kunde in seiner Not gleich zwei Emails verfasst und an Herrn Mustermann geschickt hat. So kommt es vor, dass Meier die erste Mail mit einem
anderen Ratschlag beantwortet als Schulze der sich im selben Moment der zweiten Nachricht des gleichen Kunden annimmt. Das Ergebnis ist, dass der Kunde unterschiedliche Antworten bekommt. Darüber hinaus hat Herr Mustermann keinen Einblick darüber, welcher Mitarbeiter wann was welchem Kunden gesagt hat, welche Probleme besonders häufig auftreten und wie groß sein gesamter Aufwand für den Kundensupport ist.


