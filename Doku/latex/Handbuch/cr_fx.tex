\chapter{Wie funktioniert CodeRed?}  % Kapitel % Steht dann über dem Text
\label{chapter:Wie funktioniert CodeRed?}  % Steht als Text im Inhaltsverzeichnis
\index{Wie funktioniert CodeRed?} % für das Stichwortverzeichnis

CodeRed ist ein Webbasierendes Online Ticket System. Als dieses ist das System für jeden unter einer öffentlichen Subdomäne aus dem Internet erreichbar. \\
\\
Jeder der das System Nutzen will, muss sich an dem System als erstes Registrieren. Nach erfolgreicher Anmeldung, wird der neue User direkt in sein persönliches Profil weitergeleitet. Dort kann er seine Daten Ändern oder Ergänzen. Zusätzlich hat der User die Möglichkeit in seinem Profil ein Avatar einzurichten.\footnotemark[1] \\
\\
Der neu angelegte User ist per default erst mal im System deaktiviert, er besitzt also keine weiteren Rechte. Er kann sich nur in dem System Umschauen und sein eigenes Profil Anschauen und Ändern. Um produktiv in dem System zu werden muss der neue User von einem Mentoren in seine vorgesehene Position gehoben werden. Der Mentor stuft den User als Kontakt, Betreuer oder auch als Mentor ein. Der User bekommt per E-Mail eine Benachrichtigung über die Änderung und kann sich dann mit seinem neuen Nutzerstatus am System neu Einlogen.\\ 
\\
Je nach seinem Status kann der User dann in dem System arbeiten. Die einzelnen Aufgaben und Rechte werden auf dem folgeden Seiten ausführlich erläutert.


\footnote[1]{Ein Avatar ist eine künstliche Person oder ein grafischer Stellvertreter einer echten Person in der virtuellen Welt.Avatare werden beispielsweise in Form eines Bildes, Icons oder als 3D-Figur eines Menschen oder sonst irgend eines Wesens dargestellt.}  


