\chapter{DynDNS Dienst}  % Kapitel % Steht dann über dem Text
\label{chapter:DynDNS Dienst}  % Steht als Text im Inhaltsverzeichnis
\index{DynDNS Dienst} % für das Stichwortverzeichnis

\begin{figure}[h]
\begin{center}
   \includegraphics[width=200pt]{../bilder/dyndns.png}
   \caption{Dyndns.org}
   \label{DynDNS.org}
\end{center}
\end{figure}
\textbf{Erläuterung zu DynDNS:} \\
Ein DynDNS- oder dynamischer Domain-Name-System-Eintrag bewirkt, dass ein Rechner, der eine wechselnde IP-Adresse besitzt, immer über den selben Domainnamen angesprochen werden kann.\\
\\
Ständig wechselnde Einträge sind im Domain Name System eigentlich nicht vorgesehen, stattdessen sollen Netzressourcen gespart werden, indem Einträge – oft mehrere Stunden oder sogar Tage – zwischengespeichert werden. Um nun dynamische DNS-Einträge zu ermöglichen, wird die Zeit, wie lange der Eintrag zwischengespeichert werden soll, auf das erlaubte Minimum von 60 Sekunden gesetzt.\\
\\
Um einen DynDNS-Eintrag in den Nameservern des Betreibers zu aktualisieren, wird üblicherweise ein DynDNS-Client installiert. Dies ist ein Programm, das sich automatisch bei einem IP-Wechsel mit dem DynDNS-Server verbindet und seine neue IP-Adresse übermittelt.
\\
Quelle: \href{http://de.Wikipedia.org/wiki/DynDNS}{Wikipedia: DynDNS} \\
\\
\textbf{Warum setzen wir DynDNS ein:}
DynDNS wird eingesetzt durch die etwas erschwerten Gegebenheiten beim Kunden. Der Kunde besitzt leider keine feste IP Adresse vorort und hat auch keine ausreichenden Zugriffsrechte auf seine vorhanden Domänen.\\
\\
\textbf{http://www.dyndns.com/ - Account}
\begin{table}[htbp]
\begin{center}
\begin{tabular*}{0.95\textwidth}{p{0.3\textwidth}p{0.6\textwidth}}
\hline
\textbf{Frage?} & \textbf{Auswahl} \\
\hline
Email: & CodeRed@farbspielchen.de \\
Name: & stsw \\
Passwort: & \textbf{2KkDCrcB3Q} \\
Hostname: & stsw-intern.dyndns.org \\
\hline
\end{tabular*}
\caption{DynDNS Account}
\label{table:DynDNS Account}
\end{center}
\end{table}

