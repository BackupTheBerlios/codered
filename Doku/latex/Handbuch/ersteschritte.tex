\chapter{Erste Schritte}  % Kapitel % Steht dann über dem Text
\label{chapter:Erste Schritte}  % Steht als Text im Inhaltsverzeichnis
\index{Erste Schritte} % für das Stichwortverzeichnis

Nachdem Sie sich erfolgreich am System angemeldet haben, wird die Oberfläche von CodeRed geladen. Sie befinden sich nach einer Anmeldung standardmäßig auf einer Übersichtsseite. Auf dieser Seite werden alle wichtigen Kategorien und Symbole erklärt. Je nachdem was Sie für eine Funktion in dem System erfüllen, passt sich die Hauptnavigation an. Diese Ansicht vermittelt einen schnellen Überblick über ihre verschiedenen Möglichkeiten im CodeRed System. \\
\\
\begin{figure}[h]
\begin{center}
   \includegraphics[width=450pt]{../bilder/welcome.png}
   \caption{Uebersichtsseite}
   \label{Uebersichtsseite}
\end{center}
\end{figure}
\\
\newpage
\begin{figure}[h]
   \includegraphics[width=450pt]{../bilder/Navigation.png}
   \caption{Hauptnavigation}
   \label{Hauptnavigation}
\end{figure}
Oben sehen Sie die Hauptnavigation in ihrer vollen Größe. An den Symbolen sind die Drei Kategorien des Systems klar Erkennbar. \\
\\
\begin{figure}[h]
   \centerline{\includegraphics[width=130pt]{../bilder/Profil-logos.png}}
   \caption{Rubrik -Profil}
   \label{Rubrik -Profil}
\end{figure}
\\
Jeder User des Systems hat sein persönliches Profil das er selbst Pflegen kann. Die Mentoren haben zusätzlich noch die Möglichkeit eine Liste aller Registrierten User aufzurufen. Weitere Informationen unter $>$Personen Profile,...\\
\\
\begin{figure}[h]
   \centerline{\includegraphics[width=130pt]{../bilder/ticket-logos.png}}
   \caption{Rubrik -Ticket}
   \label{Rubrik -Ticket}
\end{figure}
\\
Die Kernrubrik im CodeRed System, die eigentliche Ticketverwaltung. Weitere Informationen unter $>$Was ist ein Ticket,...
\\
\newpage
\begin{figure}[h]
   \centerline{\includegraphics[width=130pt]{../bilder/clienten-logos.png}}
   \caption{Rubrik -Clienten}
   \label{Rubrik -Clienten}
\end{figure}
Clienten sind die Auftragsorte für die Tickets. Unteranderem werden hier auch die Dokumentationen der verschieden Clienten verwaltet. Weitere Informationen unter $>$Clienten