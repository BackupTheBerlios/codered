%Kontext: Komplettes Handbuch für Codered
%Changelog:
%Timestamp			Name			Äderungen und Begründung
%

\begin{titlepage}

\title{- \textbf{CodeRed} -\\ 
\mbox{Das \textit{Trouble Ticket System}} \\
- der Staatliche Techniker Schule Weilburg  - \\
- Benutzerhandbuch -}
\author{Marco Benecke \and Jan Neuser}
\date{Ausgabe 0.1 vom \today}

\begin{figure}[h]
\begin{center}
   \includegraphics[width=219pt]{../bilder/crlogo.png}
   \label{CodeRed Logo}
\end{center}
\end{figure}

\publishers{Staatliche Techniker Schule Weilburg}

\uppertitleback{

\textbf{Version:} \\
- Projekt Dokumentation \\
- CodeRed Server Administration \\
- \underline{CodeRed Benutzer Dokumentaion} \\
% use packages: array
\begin{tabular}{ll}
Thema: & CodeRed - Handbuch - Benutzer\\
Server & Version: CodeRed RC2 Version 0306 \\
Datum: &  10.04.06\\ 
Puplic Version: & CodeRed System RC2 \\  
\end{tabular}

\vspace{2cm}

\textbf{Autoren:} \\
Benecke, Marco erreichbar unter \textit{benecke@gmail.com} \\
Neuser, Jan mit \textit{jan@truematrix.de} \\

\vspace{1cm}

\textbf{Korrekturleser:} \\
 Cristine Holzhäuser \\
 Beate Neuser \\
 Anne Neuser \\
}

\lowertitleback{Dieses Buch unterliegt den Grundgedanken der GPL und wird ausschließich mit Hilfe des Satzsystems {\LaTeX} unter Linux geschrieben. Die Veröffentlichung erfolgt als plattformunabhägiges PDF. Vielen Dank.}

\maketitle

\end{titlepage}

