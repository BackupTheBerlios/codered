\chapter{Überschrift}  % Kapitel % Steht dann über dem Text
\label{chapter:Überschrift}  % Steht als Text im Inhaltsverzeichnis
\index{Überschrift} % für das Stichwortverzeichnis

Hier wird dann der Text stehen. 
oder Auflistungen;
\begin{itemize}
\item erstes Lernziel,
\item zweites Lernziel,
\item drittes Lernziel.
\end{itemize}

Weiteres Verschachteln :

\section{Oberpunkt}
\label{section:Oberpunkt}
\index{Thema!Oberpunkt}
\index{Oberpunkt}

Gleiches gilt auch f¨ur weitere Unterpunkte.

\subsection{Unterpunkt}
\label{subsection:Unterpunkt}
\index{Oberpunkt!Unterpunkt}
\index{Unterpunkt}


Tabellen bitte einheitlich so:
Wenn nicht möglich dann anpassen.

\begin{table}[htbp]
\begin{center}
\begin{tabular*}{0.95\textwidth}{p{0.3\textwidth}p{0.6\textwidth}}
\hline
\textbf{erste Spalte} & \textbf{zweite Spalte} \\
\hline
1. & Zeile \\
2. & Zeile \\
3. & Zeile \\
\hline
\end{tabular*}
\caption{Tabellenbeschriftung}
\label{table:Tabellenbeschriftung}
\end{center}
\end{table}



So sind bilder einzubinden (Dateipfad beachten!)
\begin{figure}[htbp]
\centering
\includegraphics[width=\textwidth]{filename.eps}
\caption{Bildunterschrift}
\label{figure:Bildunterschrift}
\end{figure}


urls bitte so:
\href{http://codered.berlios.de}{Codered Projektseite}


für consolen komandos:
\verb|kommando -option parameter|

Fußnoten:
RubyonRails\footnote{eine Software- und Systemenwicklungsumgebung} für unser Projekt.
